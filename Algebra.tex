\documentclass[]{article}

\usepackage{amsmath}
\usepackage{amssymb}
\usepackage{dsfont}
\usepackage{lipsum}
\usepackage[margin=10pt]{geometry}
\setlength{\parindent}{0pt}
\setlength{\parskip}{0pt}

\begin{document}

\newcommand{\NN}{\mathds{N}}
\newcommand{\ZZ}{\mathds{Z}}
\newcommand{\PP}{\mathds{P}}
\newcommand{\Z}[1]{\ZZ/#1\ZZ}

\section*{Teilbarkeit}

$a|b\Leftrightarrow b = q\cdot a$\\
$a|a; a|0; a|b\Rightarrow a|bc$\\
$a|b,a|b'\Rightarrow a|(b+b'),a|(b-b')$\\
$a|b,b\neq 0\Rightarrow |a|\leq |b|$\\
$\forall a,b. a\neq 0, b=qa+r, 0\leq r<a$\\
$.\qquad b\mod a := r$\\
$.\qquad b\equiv b' \mod a\Leftrightarrow a|(b'-b)$\\
$ggT(a,b):= (a,b) := \max t . t|a, t|b$\\
$kgV(q,b):= [a,b] := \min v. v\neq 0, a|v, b|v$\\
$b=qa+r\Rightarrow (b,a) = (a,r)$\\
ggTAlg: \\
$a_1=q_1a_2+a_3, 0\leq a_3 < a_2$\\
$a_2=q_2a_3+a_4, 0\leq a_4 < a_3$\\
...\\
$a_{m-1} = q_{m-1}a_{m}+a_{m+1}, 0\leq a_{m+1} < a_m$\\
$a_m = q_ma_{m+1}$\\
Bezout:
$ggT(a,b)=ra+sb$\\
$t|a,t|b,a|v,b|v\Rightarrow$\\
$.\qquad (ca,cb)=c(a,b), t|(a,b),(\frac{a}{t},\frac{b}{t})=\frac{(a,b)}{t},$\\
$.\qquad [a,b]|v,[a,b]=\frac{|ab|}{(a,b)}$\\
$(t,a)=1\Rightarrow t|ab \Rightarrow t|b$\\
$n\geq2, (a,n)=1, \exists !1\leq a'< n, a\cdot a'\equiv 1\mod n$\\
$\uparrow$ ergibt sich aus $1=ra+sn$, sodass $ra\equiv 1\mod n$\\
b-adisch:$b\geq 2,n\neq 0,0\leq a_i<b$\\
$.\qquad n=a_{k-1}b^{k-1}+...a_1b^1+a_0$\\
Umwandlungsalg:\\
$a_0 := n\mod b, n_1 := (n-a_0)/b ...$\\
$Basis\ b\Rightarrow t|n\Leftrightarrow t|b$\\
Quersumme: $Q(n)=\sum a_i$\\
$Basis\ b, t|b-1 \Rightarrow t|n \Leftrightarrow t|Q(n)$\\
$A(n) = (-1)^{k-1}a_{k-1}+...-a_1+a_0$\\
$Basis\ b, t|b+1\Rightarrow t|n \Leftrightarrow A(n)$\\
Wichtiges Lemma:\\
$(x^n-y^n) = (x-y)\sum^{n-1}_{i=0}x^iy^{n-1-i}$\\
Nutzung:\\
$(b^n-1)=(b-1)(...) $ für $t|b-1\Rightarrow t|b^n-1$\\
Primzahlen:\\
$n\in\mathds{P}\Leftrightarrow\{t\ |\ t|n\} = \{1,n\}$\\
Mersennesche Primzahlen:\\
$p\in\mathds{P}, p=2^a-1$\\
$2^a-1\in\mathds{P}\Rightarrow a\in\mathds{P}$\\
Fermatsche Primzahlen:\\
$p\in\mathds{P}, p=2^a+1$\\
$2^a+1\in\mathds{P}\Rightarrow a=2^e, e\geq 0$\\
Unendliche Primzahlen Beweis:\\
$n=p_1\cdot ... \cdot p_s+1$, $p_i$ alle Primzalen. \\
Es gibt prim $p$ mit $p|n$ (Lemma). $p=p_i$.\\
$p|n,p|(n-1)\Rightarrow p|n-(n-1) \Rightarrow p|1$ wspruch.\\
$\pi(x)=|\{p\ |\ p\in\mathds{P},p\leq x\}|$\\
Primfaktorzerlegung:\\
$ n\geq 2, n=p_1^{e_1}\cdots p_s^{e_s}$\\
Kongruenzen und Äquivklassen:\\
$\ZZ/n\ZZ = \{[0],...,[n-1]\}$\\
$(a,n)=1\Rightarrow a\equiv b\mod n\Rightarrow (b,n)=1$\\
$(\Z{n})^* = \{[a]\ |\ (a,n)=1\}$\\
$[a],[b]\in(\Z{n})^*\Rightarrow [a][b]\in(\Z{n})^*$\\
$\forall [a]\in(\Z{n})^*\exists ![x]\in(\Z{n})^* [a][x]=[1]$\\
$(a,n)=1\Rightarrow [a]_n[x]_n=[c]_n$ eindeutig mod $n$\\
$ax\equiv c\mod n$ Lösung gdw. $(a,n)|c$.\\
$.\qquad x\mod\frac{n}{d}$ eindeutig. Lösungen:\\
$[x],\left[x+\frac{n}{d}\right],...,\left[x+(d-1)\frac{n}{d}\right]$\\
Chinesischer Restsatz:\\
$n_1,...,n_m$ paarweise Teilerfremd, dann\\
$x\equiv a_1\mod n_1,...,x\equiv a_m\mod n_m$\\
.$\qquad$ eindeutige Lösung $\mod n_1\cdots n_m$\\
$\varphi(n) = \#(\Z{n})^*$\\
$p$ prim, $a>0\Rightarrow\varphi(p^a)=p^a\left(1-\frac{1}{p}\right)$\\
$\varphi(nm)=\varphi(n)\varphi(m)$\\
$\varphi(n) = n\prod_{p\in\PP,p|n}\left(1-\frac{1}{p}\right)$\\
$n=n_1,...,n_m$ teilerfremd\\
$(\Z{n})^*\rightarrow(\Z{n_1})^*\rightarrow\cdots
\rightarrow (\Z{n_m})^*$\\
$[x]\mapsto ([x],...,[x])$ bijektiv und Gruppenhomo\\
(selbes ohne Stern und Ringhomo)\\

\section*{Gruppen}

$a*(b*c)=(a*b)*c; a*e=a; a*a^{-1}=e$\\
$e*a=a; a^{-1}*a=e; (a^{-1})^{-1}=a$\\
$(a*b)^{-1}=a^{-1}*b^{-1}$\\
$a*x=a*x'\Rightarrow x=x'$\\
$x*a=x'*a\Rightarrow x=x'$\\
Abelsch: $a*b=b*a$\\
Ordnung: $|G|$ oder $\#G$\\
Untergruppe $H$ von $G$:\\
$H\neq\emptyset; a,b\in H\Rightarrow ab\in H, a^{-1}\in H$\\
auch: $a,b\in H \Rightarrow ab^{-1}\in H$\\
Zyklisch: $G=\{g^n\ |\ n\in\ZZ\}=\langle g\rangle$\\
Zyklisch ist abelsch.\\
Ordnung von $g$: $\min r. g^r=e$, sonst $\infty$\\
$\langle g \rangle$ Untergruppe der Ordnung von $g$\\
Ord $G$ ist $n$, $g\in G$ Ord $n$ $\Rightarrow\langle g \rangle = G$\\
Gruppenhomomorphismus:\\
$(G,*),(H,\cdot)$ Gruppen, $f:G\rightarrow H$\\
$f(a*b)=f(a)\cdot f(b)$\\
isomorphismus: $f\circ g = id_H, g\circ f = id_G$\\
monomorph: $f$ injektiv\\
epimorph: $f$ surjektiv\\
$f$ Gruppenhom $\Rightarrow f(e_G)=e_H, f(a^{-1})=f(a)^{-1}$\\
Zyklische Gruppen gleicher Ordnung isomorph\\
Jede Zyklische Gruppe isomorph zu $\ZZ$ oder $\Z{n}$\\
Monomorph $f$, $f(g)=h \Rightarrow $Ord $g$ = Ord $h$\\
$G,H$ Gruppen. $G\times H$ Gruppe mit \\
$.\qquad(g_1,h_1)*(g_2,h_2)=(g_1g_2,h_1h_2)$\\
Wenn $(nm)=1$,:\\
$\Z{nm}\cong\Z{n}\times\Z{m}$\\
$(\Z{nm})^*\cong(\Z{n})^*\times(\Z{m})^*$\\
Endlich Erzeugt: Gruppe $G$, $A\subset G, |A|\in\NN$\\
$.\qquad$ $g\in G$ endliches Produkt aus $A$.\\
Kongruenz: $H$ untergr. $G$, $a,b\in G$,\\
$.\qquad a\equiv b\mod H :\Leftrightarrow ab^-1\in H$\\
$\rightarrow$ Äquivrel auf $G$ mit Klassen $Ha=\{ha\ |\ h\in H\}$\\
Rechtsnebenklassen isomorph.\\
Deshalb $|Ha|=|Hb|$\\
$a\equiv' b\mod H$ mit $b^{-1}a\in H$ analog\\
mit Äquivklassen $aH$\\
Lagrange: $H$ UG $G$ $\Rightarrow$ (Ord $H$)$|$(Ord $G$)\\
Index : $[G : A] = |\{Ha | a\in G\}|$\\
$[G : H] = |G|/|H|$\\
$G$ endl. Grp., $a\in G \Rightarrow$ (Ord $a$)$|$(Ord $G$)\\
Ord $G$ prim $\Rightarrow$ $G$ zyklisch\\
$G$ endl. Grp., $a\in G\Rightarrow a^{|G|}=e$\\
$n\geq 1, (a,n)=1 \Rightarrow a^{\varphi(n)}\equiv 1 \mod n$\\
$p\in\PP , p\nmid a\Rightarrow a^{p-1}\equiv 1\mod p$\\
$forall a, p\in\PP. a^p\equiv a\mod p$\\
Normalteiler:\\
$H$ UG $G$, $g\in G, h\in H\Rightarrow g^{-1}hg\in H$\\
$H\vartriangleleft G$ oder $aH = Ha$\\
$G$ abelsch, $H$ UG $G$ $\Rightarrow H\vartriangleleft G$ \\
$N\vartriangleleft G\Rightarrow $ Gruppe $G/H$ mit $Na * Nb = N(ab)$\\
$|G/N| = [G : N] = |G|/|N|$\\
Gruppenhom $f:G\rightarrow H$:\\
$Ker\ f:= \{g\in G\ |\ f(g)=e_H\}$\\
$Im\ f:= \{f(g)\ |\ g\in G\}$\\
$Ker\ f\vartriangleleft G$\\
$f$ injekt. $\Leftrightarrow Ker\ f=\{e_H\}$\\
$Im\ f$ UG $G$\\
$N\vartriangleleft G\Rightarrow N=Ker\ \pi$\\
mit $\pi : G\rightarrow G/N$ kanonisch\\
Homomorphiesatz:\\
$f:G\rightarrow H$ Gruppenhom.\\
$G/Ker\ f\cong Im\ f$\\




\end{document}