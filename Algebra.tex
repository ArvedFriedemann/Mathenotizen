\documentclass[]{article}

\usepackage{amsmath}
\usepackage{dsfont}
\usepackage{lipsum}
\usepackage[margin=10pt]{geometry}
\setlength{\parindent}{0pt}
\setlength{\parskip}{0pt}

\begin{document}

\newcommand{\NN}{\mathds{N}}
\newcommand{\ZZ}{\mathds{Z}}
\newcommand{\PP}{\mathds{P}}
\newcommand{\Z}[1]{\ZZ/#1\ZZ}

\section*{Kapitel 1}
$a|b\Leftrightarrow b = q\cdot a$\\
$a|a; a|0; a|b\Rightarrow a|bc$\\
$a|b,a|b'\Rightarrow a|(b+b'),a|(b-b')$\\
$a|b,b\neq 0\Rightarrow |a|\leq |b|$\\
$\forall a,b. a\neq 0, b=qa+r, 0\leq r<a$\\
$.\qquad b\mod a := r$\\
$.\qquad b\equiv b' \mod a\Leftrightarrow a|(b'-b)$\\
$ggT(a,b):= (a,b) := \max t . t|a, t|b$\\
$kgV(q,b):= [a,b] := \min v. v\neq 0, a|v, b|v$\\
$b=qa+r\Rightarrow (b,a) = (a,r)$\\
ggTAlg: \\
$a_1=q_1a_2+a_3, 0\leq a_3 < a_2$\\
$a_2=q_2a_3+a_4, 0\leq a_4 < a_3$\\
...\\
$a_{m-1} = q_{m-1}a_{m}+a_{m+1}, 0\leq a_{m+1} < a_m$\\
$a_m = q_ma_{m+1}$\\
Bezout:
$ggT(a,b)=ra+sb$\\
$t|a,t|b,a|v,b|v\Rightarrow$\\
$.\qquad (ca,cb)=c(a,b), t|(a,b),(\frac{a}{t},\frac{b}{t})=\frac{(a,b)}{t},$\\
$.\qquad [a,b]|v,[a,b]=\frac{|ab|}{(a,b)}$\\
$(t,a)=1\Rightarrow t|ab \Rightarrow t|b$\\
$n\geq2, (a,n)=1, \exists !1\leq a'< n, a\cdot a'\equiv 1\mod n$\\
$\uparrow$ ergibt sich aus $1=ra+sn$, sodass $ra\equiv 1\mod n$\\
b-adisch:$b\geq 2,n\neq 0,0\leq a_i<b$\\
$.\qquad n=a_{k-1}b^{k-1}+...a_1b^1+a_0$\\
Umwandlungsalg:\\
$a_0 := n\mod b, n_1 := (n-a_0)/b ...$\\
$Basis\ b\Rightarrow t|n\Leftrightarrow t|b$\\
Quersumme: $Q(n)=\sum a_i$\\
$Basis\ b, t|b-1 \Rightarrow t|n \Leftrightarrow t|Q(n)$\\
$A(n) = (-1)^{k-1}a_{k-1}+...-a_1+a_0$\\
$Basis\ b, t|b+1\Rightarrow t|n \Leftrightarrow A(n)$\\
Wichtiges Lemma:\\
$(x^n-y^n) = (x-y)\sum^{n-1}_{i=0}x^iy^{n-1-i}$\\
Nutzung:\\
$(b^n-1)=(b-1)(...) $ für $t|b-1\Rightarrow t|b^n-1$\\
Primzahlen:\\
$n\in\mathds{P}\Leftrightarrow\{t\ |\ t|n\} = \{1,n\}$\\
Mersennesche Primzahlen:\\
$p\in\mathds{P}, p=2^a-1$\\
$2^a-1\in\mathds{P}\Rightarrow a\in\mathds{P}$\\
Fermatsche Primzahlen:\\
$p\in\mathds{P}, p=2^a+1$\\
$2^a+1\in\mathds{P}\Rightarrow a=2^e, e\geq 0$\\
Unendliche Primzahlen Beweis:\\
$n=p_1\cdot ... \cdot p_s+1$, $p_i$ alle Primzalen. \\
Es gibt prim $p$ mit $p|n$ (Lemma). $p=p_i$.\\
$p|n,p|(n-1)\Rightarrow p|n-(n-1) \Rightarrow p|1$ wspruch.\\
$\pi(x)=|\{p\ |\ p\in\mathds{P},p\leq x\}|$\\
Primfaktorzerlegung:\\
$ n\geq 2, n=p_1^{e_1}\cdots p_s^{e_s}$\\
Kongruenzen und Äquivklassen:\\
$\ZZ/n\ZZ = \{[0],...,[n-1]\}$\\
$(a,n)=1\Rightarrow a\equiv b\mod n\Rightarrow (b,n)=1$\\
$(\Z{n})^* = \{[a]\ |\ (a,n)=1\}$\\
$[a],[b]\in(\Z{n})^*\Rightarrow [a][b]\in(\Z{n})^*$\\
$\forall [a]\in(\Z{n})^*\exists ![x]\in(\Z{n})^* [a][x]=[1]$\\
$(a,n)=1\Rightarrow [a]_n[x]_n=[c]_n$ eindeutig mod $n$\\
$ax\equiv c\mod n$ Lösung gdw. $(a,n)|c$.\\
$.\qquad x\mod\frac{n}{d}$ eindeutig. Lösungen:\\
$[x],\left[x+\frac{n}{d}\right],...,\left[x+(d-1)\frac{n}{d}\right]$\\
Chinesischer Restsatz:\\
$n_1,...,n_m$ paarweise Teilerfremd, dann\\
$x\equiv a_1\mod n_1,...,x\equiv a_m\mod n_m$\\
.$\qquad$ eindeutige Lösung $\mod n_1\cdots n_m$\\
$\varphi(n) = \#(\Z{n})^*$\\
$p$ prim, $a>0\Rightarrow\varphi(p^a)=p^a\left(1-\frac{1}{p}\right)$\\
$\varphi(nm)=\varphi(n)\varphi(m)$\\
$\varphi(n) = n\prod_{p\in\PP,p|n}\left(1-\frac{1}{p}\right)$\\
$n=n_1,...,n_m$ teilerfremd\\
$(\Z{n})^*\rightarrow(\Z{n_1})^*\rightarrow\cdots
\rightarrow (\Z{n_m})^*$\\
$[x]\mapsto ([x],...,[x])$ bijektiv und Gruppenhomo\\
(selbes ohne Stern und Ringhomo)\\

\section{Gruppen}




\end{document}