\documentclass[]{article}

\usepackage{amsmath}
\usepackage{dsfont}
\usepackage{lipsum}
\usepackage[margin=10pt]{geometry}
\setlength{\parindent}{0pt}
\setlength{\parskip}{0pt}

\begin{document}

\newcommand{\NN}{\mathds{N}}
\newcommand{\ZZ}{\mathds{Z}}
\newcommand{\PP}{\mathds{P}}
\newcommand{\RR}{\mathds{R}}
\newcommand{\QQ}{\mathds{Q}}

\newcommand{\infsum}{\sum^\infty_{n=0}}
\newcommand{\infsumv}[1]{\sum^\infty_{#1=0}}

\section*{Rechenregeln}
$x^nx^m=x^{n+m}$\\
$(x^n)^m = x^{nm}$\\
$x^ny^n = (xy)^n$\\
$x>y \lor x=y \lor x<y$\\
$x<y\Rightarrow -x>-y$\\
$x<y\Rightarrow x+a<y+a$\\
$x<y,a<b\Rightarrow x+a<y+b$\\
$x<y,a>0\Rightarrow ax<ay$\\
$0\leq x<y,0\leq a<b\Rightarrow ax<by$\\
$x<y,a<0\Rightarrow ax>ay$\\
$x^2>0; x>0\Rightarrow x^{-1}>0$\\
$0\leq x < y \Rightarrow x^{-1}>y^{-1}$\\
$|xy|=|x|\cdot|y|$\\
$|x+y|\leq|x|+|y|$\\
$|-x|=|x|$\\
$|\frac{x}{y}| = \frac{|x|}{|y|}$\\
$\exists \lim a_n, \lim b_n\Rightarrow \lim (a_n+b_n) = (\lim a_n) + (\lim b_n)$\\
$\exists \lim a_n, \lim b_n\Rightarrow \lim (a_n\cdot b_n) = (\lim a_n) \cdot (\lim b_n)$\\
$\exists \lim a_n, \lim b_n\Rightarrow \lim (\frac{a_n}{b_n}) = \frac{\lim a_n}{\lim b_n}$\\
$a_n\leq b_n\Rightarrow \lim a_n\leq \lim b_n$\\
$\exists \sum^\infty_{n=0}a_n, \sum^\infty_{n=0}b_n\Rightarrow  \sum^\infty_{n=0}(\lambda a_n+\mu b_n) = \lambda \sum^\infty_{n=0}a_n + \mu\sum^\infty_{n=0}b_n$\\
Bernoulli $x\geq -1, n\in\NN$: $(1+x)^n\geq 1+nx$\\
$exp(x+y) = exp(x)exp(y)$\\

\section*{Formeln}
$\binom{n}{k} = \frac{n!}{k!(n-k)!}$\\
$(x+y)^n = \sum_{k=0}^n\binom{k}{n} x^{n-k}y^{k}$\\
$\sum^n_{k=0}x^k = \frac{1-x^k}{1-x}$\\
$(1+x)^n\leq 1+nx$\\
$\lim \frac{1}{n} = 0$\\
$\lim \frac{n}{n+1} = 1$\\
$\lim \frac{n}{2^n} = 0$\\
Folge zu Reihe: $c_n=c_0+\sum^n_{k=1}(c_k-c_{k-1})$\\
$\sum_{n=0}^\infty x^n = \frac{1}{1-x}$\\
$\lim a_n = \pm\infty, a_n\neq 0 \Rightarrow \lim\frac{1}{a_n} = 0 $\\
$\lim a_n = 0, a_n>0 \Rightarrow \lim \frac{1}{a_n}=\infty$ ($-\infty$ analog $a_n<0$)\\
$\sqrt{n+p\sqrt{n+p...}}=\frac{1}{2}\left(p1+\sqrt{1+4n})\right), p\in\{1,-1\}$\\
$\sum^\infty_{n=1}\frac{1}{n^k},$  $k=1$ dann $\infty$ sonst konv.\\
TODO: Quotientenkrit.\\
$exp(x) = \sum^\infty_{n=0}\frac{x^n}{n!}$ abs. konv.\\

\section{Folgerungen}
Intervallschachtelung:\\
$I_0\subset I_1\subset...$, $\lim diam(I_n) = 0$ konv. $x\in\RR$\\
Bolzano-Weierstraß: Jede beschränkte Folge $(a_n)_{n\in\NN}$\\
 besitzt eine konvergente Teilfolge.\\
Cauchy-Konvergenz $\sum^\infty_{n=0}a_n$ konv. gdw.:\\
$\forall \epsilon > 0. \exists N.forall n,m\geq N.\left|\sum^n_{k=m}\right|<\epsilon$\\
notwendig. krit: $\lim a_n = 0$\\
konvergiert gdw. Folge Partialsummen beschränkt. \\
$a_n$ mon. fallend, nicht negativ $\lim a_n = 0$ dann\\
$\sum^\infty (-1)a^n$ konvergiert. \\
Majoranten-Krit: $\sum^\infty_{n=0}c_n$ konv.\\
$0\leq c_n\geq |a_n|\Rightarrow \sum^\infty_{n=0}|a_n|$ konv.\\
$c_n := \sum^n_{k=0}a^nb^{n-k};$
$\infsum c_n = (\infsum a_n)\cdot(\infsum b_n)$\\
.$\qquad$ wenn subsum. absol. konv.\\
$exp(x)>0; exp(-x)=\frac{1}{-x}; exp(x)=e^x$\\


\section*{Axiome}
Arch.: $x,y>0, \exists n\in\NN. nx>y$

\section*{Definitionen}
$\lim_{n\rightarrow\infty}a_n = a :\Leftrightarrow\forall \epsilon>0.\exists N . n\geq N \Rightarrow |a-a_n|<\epsilon$\\
Beschränkt: $M\geq 0, |a_n|\leq M$\\
(spezieller: oben unten)\\
$|x|<1\Rightarrow\sum^\infty_{n=0}a_n = \lim_{m\rightarrow\infty}\sum^m_{n=0}a_n$\\
Cauchy:$\forall \epsilon. \exists N.\forall n,m\leq N |a_n-a_m|<\epsilon$\\
b-adisch, $b\geq 2$: $\pm\sum^\infty_{n=-k}a_nb^{-k}$\\
Häufungspunkt: TF $a_n$ konv. geg. $a$\\
Monotonie (bsp. streng): $a_n\leq a_{n+1}$\\
Beschränkt monoton konvergiert\\
k-te Wurzel $x^k=a$:\\
$x_{n+1} = \frac{1}{k}\left((k-1)x_n+\frac{a}{x_n^{k-1}}\right)$\\
Absolute Konvergenz $\sum^\infty a_n$:\\
$\sum^\infty|a_n|$ konvergiert\\
gdw. $\sum^\infty|a_n|<\infty$.
Konv. auch im eigentlichen Sinn.\\
Berührpunkt (BP) $a$ von $A$: $U_\epsilon(a):=]a-\epsilon,a+epsilon[,\epsilon>0$\\
.$\qquad$ mit mind. einem Punkt in $A$\\
Häufungspunkt (HP): $\epsilon$-Umgeb. unendl. $A$.\\
$a\in A$ BP\\
$a_n\in A, \exists \lim a_n = a\Leftrightarrow a$ BP von $A$\\
$a$ HP $A$ $\Leftrightarrow$ $a$ BP $A/\{a\}$\\
Supremum: kleinste ob. Schranke $K$ von $A$\\
Infimum analog\\
Maximum $A$: $sup(A)\in A$ (Minimum analog)\\
Limes superior / inferior:\\
$\lim\sup a_n = \lim (\sup\{a_k : k\geq n\})$ (inf analog)\\
$\lim\sup a_n = a \Leftrightarrow$ (inferior analog)\\
.$\qquad a_n < a+\epsilon$ bis auf endl. viele\\
.$\qquad a_m > a-\epsilon$ unendlich viele $m\in\NN$\\


\end{document}